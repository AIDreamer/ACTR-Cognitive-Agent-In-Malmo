\documentclass{beamer}
\usepackage{beamerthemeCogWorks}

\beamertemplatenavigationsymbolsempty

\usepackage{tikz}
\usetikzlibrary{arrows}
\usetikzlibrary{patterns}
\usetikzlibrary{positioning}
\usetikzlibrary{shapes}

\pgfdeclarelayer{background}
\pgfdeclarelayer{foreground}
\pgfsetlayers{background,main,foreground}

\usepackage{multirow}

\usepackage{listings}
\usepackage{minted}

\begin{document}

\title{The JSON Network Interface (JNI) Module}
%\title{A remote ACT-R device API}
\subtitle{Connecting ACT-R to the World with JSON over TCP}
\author{Ryan M. Hope}
\date{\today}

\frame{\titlepage}

\frame{\frametitle{Device Interfaces}
In ACT-R, a device interface represents the external world/environment.
\begin{itemize}
\item Creating an ACT-R device can be harder than writing an ACT-R model. 
\item They must be written in Lisp
\item Common Lisp has limited solutions for making GUIs and graphical applications
\item Devices are often ``simulations'' of more complicated environments
\end{itemize}
}

\frame{\frametitle{Benefits of the JNI}
\begin{itemize}
\item Not one line of Lisp needs to be written!
\item Works with every programming language
\item Works on all operating systems
\item TCP based communication allows ACT-R and environment can be run on separate machines
\end{itemize}
}

\frame{\frametitle{Design}
\tikzstyle{operator}=[rectangle, draw, fill=blue!20, rounded corners, text width=8em, minimum height=6em, text centered, execute at begin node=\Large]
\tikzstyle{visual}=[operator, fill=green!20]
\tikzstyle{motor}=[operator, fill=red!20]
\tikzstyle{sc}=[operator, fill=green!20]

\begin{center}
\begin{tikzpicture}[node distance=16em,auto,>=latex']
	\node (jni) [visual] {JNI Module\\(Client)};
	\node (env) [motor, xshift=-.2em, right of=jni] {Environment\\(Server)}
		edge[<->,very thick] node[auto,above] {JSON API} (jni);
\end{tikzpicture}
\end{center}

}

%\frame{\frametitle{Why is a ``remote device API'' needed?}
%\begin{alertblock}{}
%To standardize how ACT-R communicates with non-lisp based environments.
%\end{alertblock}
%}

\begin{frame}[fragile]
\frametitle{What is JSON}
JSON (JavaScript Object Notation) is a lightweight data-interchange format
\begin{itemize}
\item is easy for humans to read and write
\item it is easy for machines to parse and generate
\item it is built on two structures, \textit{objects} and \textit{arrays}
\end{itemize}
\vfill
Example:
\begin{center}\scriptsize
\begin{minted}{json}
{"keyA": [1, 2], "keyB": false, "keyC": true, "keyD": null}
\end{minted}
\end{center}
\end{frame}

\begin{frame}[fragile]
\frametitle{The JSON API}
\small{
\begin{minted}{json}
{"model":<modelID>,"method":<command>,"params":<params>}\r\n
\end{minted}
}
\vfill
\begin{columns}
	\scriptsize
	\begin{column}{.5\linewidth}
		JNI Commands:
		\begin{itemize}
		\item keypress
		\item mousemotion
		\item mouseclick
		\item speak
		\item reset
		\item model-run
		\item model-stop
		\item set-mp-time
		\item gaze-loc
		\item attention-loc
		\end{itemize}
	\end{column}
	\begin{column}{.5\linewidth}
		Environment Commands:
		\begin{itemize}
		\item sync
		\item setup
		\item update-display
		\item trigger-reward
		\item trigger-event
		\item set-cursor-loc
		\item new-digit-sound
		\item new-tone-sound
		\item new-word-sound
		\item new-other-sound
		\end{itemize}
	\end{column}
\end{columns}
\end{frame}

\begin{frame}[fragile]
\frametitle{Example ``update-display'' command}

\small{
\begin{minted}{json}
{"model": "myModel",
 "method": "update-display",
 "params": {
  "visual-location-chunks": [
   {"isa": "visual-location", 
    "slots": {"screen-x": 100,
              "screen-y": 200}},
   {"isa": "visual-location",
    "slots": {"screen-x": 300,
              "screen-y": 400}}],
  "visual-object-chunks": [ 
    [{"isa": "visual-object",
      "slots": {"value": "hello"}},
     {"isa": "visual-object",
      "slots": {"value": ":world"}}]],
  "clear": true}}
\end{minted}
}
\end{frame}


\frame{\frametitle{Configuring the Module}
\begin{center}
\texttt{(sgp :jni-hostname "localhost" :jni-port 5454)}
\end{center}
}

\frame{\frametitle{Where to get the JNI}
\texttt{https://github.com/RyanHope/json-network-interface}
\vfill
\textbf{Everything is on GitHub:}
\begin{itemize}
\item Source code
\item Links to binaries
\item Installation instructions
\item Documentation
\item Examples
\item Issue Tracker (For bugs \& feature requests)
\end{itemize}
}

\frame{\frametitle{High Level Libraries}
\textbf{actr6{\_}jni} - A Python$/$Twisted for communication with the JNI module, available in the Python Package Index (PyPi)
\vfill
More to come!
}

\end{document}